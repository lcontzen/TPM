%% \documentclass[xcolor=dvipsnames]{beamer}
\documentclass[pdf]{beamer}
\mode<presentation>{\usetheme{Warsaw}}
%  \usecolortheme{seagull}}
\usepackage[utf8]{inputenc}
\usepackage[french]{babel}
\usepackage[T1]{fontenc}
\usepackage{verbatim}

\title[Interrogation de bases de données via un langage naturel]{Interrogation de bases de données via un langage naturel}
\subtitle{Application au domaine de l'édition}
\author[Laurent Contzen]{Laurent Contzen}
\date{2012-2013}

\begin{document}
\frame{\titlepage}

\frame{\tableofcontents}
\section{Introduction}
\begin{frame}
  \begin{center}
    \structure{\Huge \insertsection}
  \end{center}
\end{frame}
\subsection{Intérêt du sujet}
\frame{\frametitle{\insertsubsection}
  \begin{itemize}
  \item<1-> Multidisciplinaire
    \begin{itemize}
    \item<2-> Bases de données
    \item<3-> Ingénierie Linguistique
    \item<4-> Programmation
    \item<5-> Information et Société
    \end{itemize}
  \item<6-> Grandes possibilités d'applications
  \end{itemize}
}

\subsection{Domaine d'application choisi}
\frame{\frametitle{\insertsubsection}
  \begin{itemize}
  \item<1-> Domaine de l'édition, plus particulièrement des films
  \item<2-> Beaucoup de données existantes
  \item<3-> Données semi-structurées
  \item<4-> Domaine répandu
  \item<5-> Possibilités de recruter des connaisseurs pour la phase de test
  \end{itemize}
}

\section{Méthodologie}
\begin{frame}
  \begin{center}
    \structure{\Huge \insertsection}
  \end{center}
\end{frame}

\frame{\frametitle{\insertsection}
  \begin{itemize}
  \item<1-> Étude des travaux déjà effectués dans les différents domaines
  \item<2-> Mise en relation des différents aspects du mémoire
  \item<3-> Modélisation du modèle d'application et étude des possibilités de jeux de requêtes
  \item<3-> Implémentation
  \item<4-> Phase de test
  \item<5-> Analyse approfondie des résultats selon différents points de vue
  \end{itemize}
}

%% \section{Table des matières}
%% \begin{frame}
%%   \begin{center}
%%     \structure{\Huge \insertsection}
%%   \end{center}
%% \end{frame}

%% \frame{\frametitle{\insertsection}
%%   \begin{itemize}
%%   \item<1-> Introduction
%%   \item<2-> État de l'art
%%   \item<3-> Modélisation du domaine d’application choisi
%%   \item<4-> Étude des possibilités de jeux de requêtes
%%   \item<5-> Construction de la base de données et de son contenu 
%%     \begin{itemize}
%%     \item<6-> Conception du schéma relationnel de la base de données 
%%     \item<7-> Collecte et insertion de données dans la base de données
%%     \end{itemize}
%%   \end{itemize}
%% }

%% \frame{\frametitle{\insertsection ~- Suite}
%%   \begin{itemize}
%%   \item<1-> Conception des requêtes et des réponses
%%     \begin{itemize}
%%     \item<2-> Conception de l’analyse des requêtes
%%     \item<3-> Conception de la recherche de réponse
%%     \item<4-> Conception de la méthode de formulation de réponses
%%     \end{itemize}
%%   \item<5-> Implémentation
%%   \item<6-> Évaluation des résultats
%%     \begin{itemize}
%%     \item<7-> Méthode
%%     \item<8-> Résultats
%%     \end{itemize}
%%   \item<9-> Conclusion
%%   \end{itemize}
%% }

\section{Conclusion}
\begin{frame}
  \begin{center}
    \structure{\Huge \insertsection}
  \end{center}
\end{frame}

\end{document}

% LocalWords:  semi-structurées Implémentation
