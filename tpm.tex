\documentclass[11pt]{article}
\usepackage[utf8]{inputenc}
\usepackage[french]{babel}
\usepackage[T1]{fontenc}
\usepackage{verbatim}
\usepackage{csquotes}
\usepackage{graphicx}
\usepackage{fullpage}
\usepackage{hyperref}
\usepackage[usenames,dvipsnames]{color}
\usepackage{textcomp}
\usepackage{fix-cm}
\usepackage{float}
\usepackage{biblatex}
\usepackage{setspace}
\usepackage[table]{xcolor}
\definecolor{light-gray}{gray}{0.75}
\definecolor{ulbleu}{rgb}{0, .3, .57}
\hypersetup{
  pdftitle={Interrogation de bases de données en langage naturel – Application au domaine de l’édition. Travail Préparatoire au Mémoire},
  pdfauthor={Contzen Laurent},
  colorlinks=true,
  linkcolor=ulbleu,
  urlcolor=ulbleu,
  citecolor=ulbleu,
  linktoc=all
}
\usepackage{booktabs}
\usepackage{listings}
\author{Contzen Laurent}
\bibliography{tpm}
\onehalfspacing
\begin{document}

\begin{titlepage}  
  \begin{flushleft}
    Contzen Laurent
  \end{flushleft}
  \begin{center}
    \vspace{82mm}\LARGE{\textbf{Interrogation de bases de données en langage naturel – Application au domaine de l’édition.} \\    
      Travail Préparatoire au Mémoire}
  \end{center}
  \begin{flushright}
    \vspace{92mm}
    Année Académique 2011-2012.             
  \end{flushright}
\end{titlepage}

\tableofcontents
\newpage

\section{Introduction}
L'action d'effectuer des requêtes dans un système d'information requiert souvient d'utiliser un langage contrôlé, de structurer les requêtes sous une cetaine forme, d'utiliser des formulaires ou des opérateurs booléens directement et encore bien d'autres contraintes. Ceci pose de nombreux problèmes d'énonciation ou de compréhension de requêtes. Pouvoir dialoguer de manière naturelle dans un langage naturel avec un système d'information permettrait de résoudre beaucoup de ces problemes. \colorbox{BrickRed}{\textbf{TODO : } Parler de DBs} \\
L'objectif du mémoire faisant ici l'objet de ce travail préparatoire est d'essayer d'implémenter un logiciel permettant à un utilisateur de poser une question en langage naturel et d'obtenir, s'il en existe une, une réponse dans le même langage naturel. Ceci a pour volonté d'essayer de déterminer s'il est possible de dialoguer de manière conceptuelle avec une base de données.

\section{Table des matières commentée du mémoire}
\subsection{Introduction}
\subsection{État de l'art}
Avant de démarrer l'étude et l'implémentation d'une nouvelle solution, il est important d'étudier les systèmes déjà en place, leurs historiques respectifs ainsi que les technologies existantes.
\subsection{Conception des schémas de la base de données}
Les schémas de la base de données joueront un rôle primordial, l'étude de leurs conceptions est donc primordiale.
\subsection{Méthode de peuplement de la base de données}
Une fois les schémas de la base de données mis en place il sera temps de peupler ces dernières. La récupération d'informations et l'insertion de ces dernières dans la base de données joueront un rôle crucial par rapport aux résultats de la phase de test.
\subsection{Conception de l'analyse des requêtes des utilisateurs}

\subsection{Implémentation}
\subsection{Etablissement de la méthode d'évaluation du logiciel}
\subsection{Résultats de l'évaluation}
\subsection{Conclusion}
\subsection{Bibliographie}


\section{Bibliographie commentée}

\section{Chapitre introductif}

\section{Plan de travail}

\section{Description du projet de mémoire}
Implémentation d’un logiciel ayant pour but de permettre  à un utilisateur de poser une question en langage naturel et d’obtenir, s’il en existe une, une réponse dans le même langage naturel. La question sera traitée afin d’en extraire les informations utiles et ces dernières seront ensuite confrontées à une base de données dont l’architecture sera l’objet d’une étude détaillée. Les informations extraites de cette base de données seront ensuite reformulées en langage naturel. Le domaine d’application sera celui des livres et/ou des films. Ce mémoire mêlera donc ingénierie linguistique et bases de données.

\section{Conclusion}

\newpage
\section{Sources}
\nocite{*}
\printbibliography

\end{document}
