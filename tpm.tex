\documentclass[11pt]{article}
\usepackage[utf8]{inputenc}
\usepackage[french]{babel}
\usepackage[T1]{fontenc}
\usepackage{verbatim}
\usepackage{csquotes}
\usepackage{graphicx}
\usepackage{fullpage}
\usepackage{hyperref}
\usepackage[usenames,dvipsnames]{color}
\usepackage{textcomp}
\usepackage{fix-cm}
\usepackage{float}
\usepackage{biblatex}
\usepackage{setspace}
\usepackage[table]{xcolor}
\definecolor{light-gray}{gray}{0.75}
\definecolor{ulbleu}{rgb}{0, .3, .57}
\hypersetup{
  pdftitle={Interrogation de bases de données en langage naturel – Application au domaine de l’édition. Travail Préparatoire au Mémoire},
  pdfauthor={Contzen Laurent},
  colorlinks=true,
  linkcolor=ulbleu,
  urlcolor=ulbleu,
  citecolor=ulbleu,
  linktoc=all
}
\usepackage{booktabs}
\usepackage{listings}
\defbibheading{bibliography}{\section{Bibliographie}}
\author{Contzen Laurent}
\bibliography{tpm}
\onehalfspacing
\begin{document}

\begin{titlepage}  
  \begin{flushleft}
    Contzen Laurent
  \end{flushleft}
  \begin{center}
    \vspace{82mm}\LARGE{\textbf{Interrogation de bases de données en langage naturel – Application au domaine de l’édition.} \\    
      Travail Préparatoire au Mémoire}
  \end{center}
  \begin{flushright}
    \vspace{92mm}
    Année Académique 2011-2012.             
  \end{flushright}
\end{titlepage}

\tableofcontents
\newpage

\section{Introduction}
L'action d'effectuer des requêtes dans un système d'information requiert souvient d'utiliser un langage contrôlé, de structurer les requêtes sous une cetaine forme, d'utiliser des formulaires ou des opérateurs booléens directement et encore bien d'autres contraintes. Ceci pose de nombreux problèmes d'énonciation ou de compréhension de requêtes. Pouvoir dialoguer de manière simplifiée avec un système d'information permettrait de résoudre beaucoup de ces problemes. \colorbox{BrickRed}{\textbf{TODO : } Parler de DBs} \\
L'objectif du mémoire faisant ici l'objet de ce travail préparatoire est d'essayer d'implémenter un logiciel permettant à un utilisateur de poser une question en langage naturel et d'obtenir, s'il en existe une, une réponse dans le même langage naturel. Ceci a pour volonté d'essayer de déterminer s'il est possible de dialoguer de manière conceptuelle avec une base de données.

\section{Table des matières commentée du mémoire}
\subsection{Introduction}
\subsection{État de l'art}
Avant de démarrer l'étude et l'implémentation d'une nouvelle solution, il est important d'étudier les systèmes déjà en place, leurs historiques respectifs ainsi que les technologies existantes.
\subsection{Conception des schémas de la base de données}
Les schémas de la base de données joueront un rôle primordial, l'étude de leurs conceptions est donc essentielle.
\subsection{Méthode de peuplement de la base de données}
Une fois les schémas de la base de données mis en place il sera temps de peupler ces dernières. La récupération d'informations et l'insertion de ces dernières dans la base de données joueront un rôle crucial par rapport aux résultats de la phase de test.
\subsection{Conception de l'analyse des requêtes des utilisateurs}

\subsection{Implémentation}
\subsection{Conception de la méthode d'évaluation du logiciel}
\subsection{Résultats de l'évaluation}
\subsection{Conclusion}
\subsection{Bibliographie}


\section{Bibliographie commentée}

\section{Chapitre introductif}

\section{Plan de travail}

\section{Notes variées}
Cette section ne sera pas dans le travail final mais la plupart des informations contenues ici se retrouveront probablement autre part dans le travail.
\subsection{Partie Ingénierie Linguistique}
Après un rendez vous avec Max De Wilde afin de préparer cette partie, nous sommes arrivés aux conclusions suivantes :
\begin{itemize}
  \item IMDB offre la possibilité de télécharger le contenu de leur base de données en plain text et il existe plusieurs sites offrant une API permettant d'interagir avec ces données. Se limiter aux films dans un premier temps comme domaine d'application semble déjà conséquent et suffisant.
  \item Il y a pas mal d'informations interessantes sur le sujet ou pouvant être liées au sujet dans le cours en ligne actuellement donné par l'université de stanford (NLP class), c'est une bonne source.
  \item On partirait sur deux méthodes d'interrogation de la base de données pour l'utilisateur. La première utiliserait un formalisme que l'utilisateur devrait respecter (par exemple actor=''<nom>'') alors que la seconde serait plus libre. Il sera alors possible et interessant de comparer les résultats des deux méthodes. Le formalisme serait bien évidement plus intuitif que les techniques standard d'interrogation de bases de données ou de recherche documentaire actuelles.
  \item Il faudra bien définir les types d'entités que l'on désire reconnaître dans les questions. Partir sur un base de Noms de films, noms de personnes, et années parait être un bon début.
  \item Il faudrait réussir à différencier les acteurs, réalisateurs, personnages, etc.
  \item Le formalisme permettra clairement une granularité de recherche plus précise.
  \item Il faudra définir un ordre d'application dans le mapping.
  \item Il est envisageable d'utiliser l'API d'OpenCalais afin de vérifier certains résultats. Malheureusement contrairement au nom ce service n'est pas open-source.
  \item Python est le langage de choix à utiliser pour cette partie du mémoire. Le framework NLTK est toutefois parfois fort lourd, surtout pour ce qui sera à faire. Il faudra donc bien réfléchir s'il est nécéssaire de l'utiliser ou s'il est possible de faire le nécéssaire sans.
\end{itemize}

\subsection{Description du projet de mémoire}
Implémentation d’un logiciel ayant pour but de permettre  à un utilisateur de poser une question en langage naturel et d’obtenir, s’il en existe une, une réponse dans le même langage naturel. La question sera traitée afin d’en extraire les informations utiles et ces dernières seront ensuite confrontées à une base de données dont l’architecture sera l’objet d’une étude détaillée. Les informations extraites de cette base de données seront ensuite reformulées en langage naturel. Le domaine d’application sera celui des livres et/ou des films. Ce mémoire mêlera donc ingénierie linguistique et bases de données.

\section{Conclusion}

\newpage
\nocite{*}
\printbibliography

\end{document}
